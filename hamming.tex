\subsection{Код Хэмминга}
%\addcontentsline{toc}{subsection}{Код Хэмминга}

Целью данной работы является построение декодера на основе нейронной сети. Для её обучения требуется достаточный набор данных. Код Хэмминга $\mathbb{C}(n,k)$, $n = 16, k = 11$, подходит для обучения, так как пространство его информационных слов достаточно для~обучения нейронной сети.

Порождающая матрица $G$ имеет вид:
\begin{equation}
G =
\begin{pmatrix}
0 & 0 & 0 & 0 & 0 & 0 & 0 & 0 & 1 & 0 & 0 & 1 & 1 & 0 & 1 & 0 \\
0 & 0 & 0 & 0 & 0 & 0 & 1 & 0 & 0 & 0 & 0 & 0 & 1 & 0 & 1 & 1 \\
0 & 0 & 0 & 0 & 0 & 1 & 0 & 0 & 0 & 0 & 0 & 0 & 1 & 1 & 0 & 1 \\
0 & 0 & 0 & 0 & 1 & 0 & 0 & 0 & 0 & 0 & 0 & 0 & 1 & 1 & 1 & 0 \\
0 & 0 & 0 & 1 & 0 & 0 & 0 & 0 & 0 & 0 & 0 & 1 & 0 & 0 & 1 & 1 \\
0 & 0 & 1 & 0 & 0 & 0 & 0 & 0 & 0 & 0 & 0 & 1 & 0 & 1 & 0 & 1 \\
0 & 1 & 0 & 0 & 0 & 0 & 0 & 0 & 0 & 0 & 0 & 1 & 0 & 1 & 1 & 0 \\
1 & 0 & 0 & 0 & 0 & 0 & 0 & 0 & 0 & 0 & 0 & 1 & 1 & 0 & 0 & 1 \\
0 & 0 & 0 & 0 & 0 & 0 & 0 & 1 & 0 & 0 & 0 & 0 & 0 & 1 & 1 & 1 \\
0 & 0 & 0 & 0 & 0 & 0 & 0 & 0 & 0 & 1 & 0 & 1 & 1 & 1 & 0 & 0 \\
0 & 0 & 0 & 0 & 0 & 0 & 0 & 0 & 0 & 0 & 1 & 1 & 1 & 1 & 1 & 1
\end{pmatrix}
\end{equation}

Пусть $L = 2^k$. Кодирование информационных слов из $\mathbb{F}_2^L$ в код $\mathbb{C}$ осуществляется путем умножения на матрицу $G$.
Процесс кодирования заключается в~умножении информационного слова на~данную матрицу.

На практике доказано, что при~систематическом виде нейронная сеть вырождается в~тождественный оператор. Именно этим обусловлен выбор такой порождающей матрицы.

Проверочные биты записываются в~конец информационного слова. Для нейронной сети не~имеет значения, находятся ли~проверочные биты целиком в~конце или на~определенных позициях слова.
%\newpage
Для~проверки корректного информационных слов на наличие ошибок используется соответствующая проверочная матрица.

\begin{equation}
H =
\begin{pmatrix}
1 & 1 & 1 & 1 & 0 & 0 & 0 & 0 & 1 & 1 & 1 & 1 & 0 & 0 & 0 & 0 \\
1 & 0 & 0 & 0 & 1 & 1 & 1 & 0 & 1 & 1 & 1 & 0 & 1 & 0 & 0 & 0 \\
0 & 1 & 1 & 0 & 1 & 1 & 0 & 1 & 0 & 1 & 1 & 0 & 0 & 1 & 0 & 0 \\
0 & 1 & 0 & 1 & 1 & 0 & 1 & 1 & 1 & 0 & 1 & 0 & 0 & 0 & 1 & 0 \\
1 & 0 & 1 & 1 & 0 & 1 & 1 & 1 & 0 & 0 & 1 & 0 & 0 & 0 & 0 & 1
\end{pmatrix}
\end{equation}

При~таких параметрах данный код позволяет гарантированно исправлять одну ошибку.

%   \newpage 