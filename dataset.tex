\section{Набор данных}
%\addcontentsline{toc}{section}{Набор данных}

Набор данных является одним из~важнейших объектов, лежащих в~основе обучения нейронной сети\cite[с.\,23]{bib:neural_networks}. Для~возможности обучения и проверки качества работы модели необходимо иметь репрезентативный\cite{bib:representative_data_sets} набор кодовых слов без~ошибок и с~ошибками.
Сгенерируем все информационные слова из $\mathbb{F}_2^L$. Получим множество информационных слов $A = \{a_0; a_1; \dots; a_{L}\}$. Элементы множества $A$ имеют следующий вид:
\begin{gather}
 % Remove numbering (before each equation)
  \nonumber a_0 = (0, 0, 0, 0, 0, 0, 0, 0, 0, 0, 0), \\
  \nonumber a_1 = (0, 0, 0, 0, 0, 0, 0, 0, 0, 0, 1), \\
  \nonumber \cdots \\
  \nonumber a_L = (1, 1, 1, 1, 1, 1, 1, 1, 1, 1, 1).
\end{gather}

Получим множество всех кодовых слов:
\begin{equation*}
  \mathbb{C} = \{a\cdot G~|~\forall a \in A \}
\end{equation*}

Получим $|\mathbb{C}| = L, |A| = L$. Кодовые слова кода $\mathbb{C}$ имеют следующий вид:
\begin{gather}
 % Remove numbering (before each equation)
    \nonumber c_0 = (0, 0, 0, 0, 0, 0, 0, 0, 0, 0, 0, 0, 0, 0, 0), \\
    \nonumber c_1 = (0, 0, 0, 0, 0, 0, 0, 0, 0, 0, 1, 1, 1, 1, 1), \\
    \nonumber \cdots \\
    \nonumber c_L = (1, 1, 1, 1, 1, 1, 1, 1, 1, 1, 1, 1, 1, 1, 1).
\end{gather}
\newpage
Для~генерации кодовых слов с~ошибками создадим множество масок ошибок $E = \{e_0; \dots; e_n\}.$
\begin{gather}
 % Remove numbering (before each equation)
    \nonumber e_0 = (0, 0, 0, 0, 0, 0, 0, 0, 0, 0, 0, 0, 0, 0, 0), \\
    \nonumber e_1 = (1, 0, 0, 0, 0, 0, 0, 0, 0, 0, 0, 0, 0, 0, 0), \\
    \nonumber e_2 = (0, 1, 0, 0, 0, 0, 0, 0, 0, 0, 0, 0, 0, 0, 0), \\
    \nonumber \cdots                                                \\
    \nonumber e_n = (0, 0, 0, 0, 0, 0, 0, 0, 0, 0, 0, 0, 0, 0, 1).
\end{gather}

Определим множество кодовых слов с ошибкой $\widetilde{\mathbb{C}_i}$.
\begin{equation}
  \widetilde{\mathbb{C}_i} = \{c_i \oplus e ~|~  c_i \in \mathbb{C}~\space~\forall e \in E \}, где i \in \{0; \dots; L\}
\end{equation}
где $\oplus$ - операция сложения векторов по модулю $2$. Обозначим его элементы $\widetilde{c_i}$, где $i \in \{0; \dots; n\}$ - позиция единицы в образующей маске ошибки.

Определим множество всевозможных кодовых слов с ошибкой.
\begin{equation}
  \widetilde{\mathbb{C}} = \bigcup\limits_{i=0}^{L} \widetilde{\mathbb{C}_i}
\end{equation}

Набор данных $D$ состоит из~$L$ блоков, каждый из которых состоит из $n+1$ строк и определяется случайно выбранным кодовым словом $c~\in~\mathbb{C}$.
Блок $B_i$, определяемый кодовым словом $c_i \in \mathbb{C}$, содержит множество всевозможных пар $c_i$ и соответствующих ему  слов $\widetilde{c} \in \widetilde{\mathbb{C}_i}$:
\begin{equation}
  B_i = \{ (c_i, \widetilde{c}) ~|~ \forall~\widetilde{c} \in \widetilde{\mathbb{C}_i} \}, i \in \{0; \dots; L\}
\end{equation}

Перед началом обучения нейронной сети набор данных $D$ перемешивается и сбалансированно делится на~тренировочную и валидационную выборки $D_T$ и $D_V$ соответственно. Отношение тренировочной выборки к~валидационной определяется в главе ХХХ.
\newpage
\centerline{\textbf{Пример нескольких строк набора данных}}
Ниже приведены некоторые строки из набора данных: \\
\begin{eqnarray*}
% \nonumber % Remove numbering (before each equation)
&(0010100000100100,\>0010100000110100)& \\
&(0010100000100100,\>0010100000100000)& \\
&(0010100000100100,\>0010100010100100)& \\
&(0010100000100100,\>0010100001100100)& \\
&(0010100000100100,\>0000100000100100)& \\
&(0010100000100100,\>0010101000100100)& \\
&(0010100000100100,\>0110100000100100)& \\
&(0010100000100100,\>0010100000100110)& \\
&(0010100000100100,\>0010100000101100)& \\
&(0010100000100100,\>0010100000000100)& \\
&(0010100000100100,\>0011100000100100)& \\
&(0010100000100100,\>0010100000100100)& \\
&(0010100000100100,\>0010110000100100)& \\
&(0010100000100100,\>0010000000100100)& \\
&(0010100000100100,\>0010100100100100)& \\
&(0010100000100100,\>1010100000100100)&
\end{eqnarray*}

\newpage 