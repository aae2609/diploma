\section{Введение}
%\addcontentsline{toc}{section}{Введение}

В~последние годы методы глубокого обучения продемонстрировали значительные улучшения в~различных задачах\cite{bib:buhrstein}. Эти методы превосходят человека в~решении задачи обнаружения объектов и позволяют достичь самых высоких результатов в~области машинного перевода и обработки речи. Кроме того, глубокое обучение в~сочетании с~методом обучения с~подкреплением выигрывают партии у~чемпионов в~сложных играх, например в~Го\cite[c.\,484–489]{bib:win_go}.

Помехоустойчивые коды\cite[с.\,4]{bib:codes} используются для~установки надежного соединения и позволяют обмениваться информацией и исправлять ошибки, полученные в~результате передачи данных по каналу связи. Одним~из~примеров такого кода является код~Хэмминга\cite[с.\,49]{bib:codes2}. За~счет добавления к~информационному слову избыточности, он~позволяет исправлять одну ошибку. Декодирование\cite[c.\,5]{bib:codes} кода можно представить в~виде некоторой функции. Согласно универсальной теореме аппроксимации\cite{bib:approximation_theorem} нейронная сеть\cite[с.\,23]{bib:neural_networks} может аппроксимировать непрерывную функцию многих переменных с~любой точностью.

В~данной работе представлено исследование возможности аппроксимации нейронной сетью $\mathcal{N}$ функции декодирования кода Хэмминга~(16,~11). Прямым подходом к~решению задачи декодирования кода Хэмминга нейронной сетью является обучение модели на~наборе данных, состоящем из большого количества кодовых слов с~добавлением ошибки в~случайно выбранной позиции.


\newpage 