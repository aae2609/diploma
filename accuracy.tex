\subsection{Качество работы декодера}\label{subsec:accuracy}

Определим функцию $eq$, принимающую некоторый вектор $a$ и некоторый вектор $b$ такие, что $|a|~=~|b|$:
\begin{equation}\label{eq:equal}
    eq(a, b) =
    \begin{cases}
        1, & \mbox{если } a~=~b, \\
        0, & \mbox{иначе}.
    \end{cases}
\end{equation}

Определим функцию $A(\mathcal{N}, D_V)$:
\begin{equation}\label{eq:accuracy}
    A(\mathcal{N}, D_V) = \frac{\sum\limits_{(c, \widetilde{c})\in D_V}eq(c, \mathcal{N}(\widetilde{c}))}{|D_V|}
\end{equation}

Проверка качества работы нейронной сети $\mathcal{N}$ на валидационной выборке $D_V$ осуществляется посредством функции $A$. Для нейронной сети $\mathcal{N}$ с параметрами $s^*$:
\begin{equation}\label{eq:final_accuracy}
  A(\mathcal{N}_{s^*}, D_V)~\approx~0.92.
\end{equation}

Таким образом, получен декодер кода Хэмминга(16,\,11) на основе нейронной сети с вероятностью исправления одной ошибки, равной $92 \%$.
