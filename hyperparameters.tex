\subsection{Гиперпараметры}\label{subsec:hyperparameters}

\begin{definition}
  Гиперпараметры - это такие параметры нейронной сети, которые не изменяются в~процессе обучения нейронной сети, но от~выбора которых зависит последующее качество работы сети.
\end{definition}

Определим следующий список гиперпараметров.
\begin{itemize}
  \item Количество эпох обучения $\mathbb{\varepsilon}$ нейронной сети,
  \item Количество слоев $L \in \mathbb{N}$ нейронной сети,
  \item Количество нейронов $K_i \in \mathbb{N}$ в каждом слое $l_i, i \in [1, L]$,
  \item Функция активации\cite[раздел activations]{bib:keras} нейронов $f_{a,i}: \mathbb{R} \rightarrow \mathbb{R}$ в каждом слое $l_i, i \in [1, L]$ ,
  \item Размер батча $Z \in \mathbb{N}$ [ссылка],
  \item Функция потерь\cite[раздел losses]{bib:keras} $f_{loss}$.
\end{itemize}

Для достижения максимально высокого качества работы сети необходимо найти оптимальный набор значений гиперпараметров.
Для поиска таких параметров используется библиотека hyperopt\cite{bib:hyperopt} для языка Python.
Она содержит набор функций, удобных для поиска наилучшего набора гиперпараметров.
\newpage
Определим множества значений гиперпараметров, среди которых будет осуществляться выбор.
\begin{equation}\label{eq:hyper1}
  \mathbb{\varepsilon} \in [10, 10000]
\end{equation}

\begin{equation}\label{eq:hyper1}
  L \in \{3; 4; 5; 6; 7\}
\end{equation}

\begin{equation}\label{eq:hyper2}
  K_i \in \{64; 128; 256; 512; 1024; 2048\}, где i \in \{1; \dots; L\}
\end{equation}

\begin{equation}\label{eq:hyper3}
  Z \in \{8; 16; 32; 64; 128; 256; 512\}
\end{equation}

\begin{equation}\label{eq:hyper4}
  f_{a,i} \in \{f_{relu}; f_{tanh}; f_{softmax}; f_{sigmoid}; f_{softplus}; f_{elu}\}, i \in \{1; \dots; L\}
\end{equation}

\begin{equation}\label{eq:hyper4}
  f_{loss} \in \{f_{mse}; f_{mae}; f_{binary\_crossentropy}; f_{log}; f_{exp}\}
\end{equation}

Таким образом, декартово произведение множеств значений гиперпараметров образует пространство $\mathbb{S}$.
%Adam - Оптимизация параметров производится методом стохастического градиентного спуска по дискретному множеству параметров. При этом на каждом шаге производится обучение нейронной сети и фиксация результатов.

%С ее помощью был выбран набор, определяющий количество слоев и нейронов в каждом из них, функции активации, оптимизатор, функцию потерь, размер батча и количество эпох.
\newpage 